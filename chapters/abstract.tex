% !TeX root = ../main.tex

\begin{abstract}
  	摘要是论文的内容不加注释和评论的简短陈述,应具有独立性和自含性,即不阅读论文全文,就能获得必要的信息。摘要中有数据、有结论,是一篇完整的短文,可以独立使用,可以引用,可以用于工艺推广。摘要的内容应包含与论文同等量的主要信息,供读者确定有无必要阅读全文,也供文摘等二次文献采用。摘要一般应说明研究工作目的、实验方法、结果和最终结论等,而重点是结果和结论。
	硕士学位论文中中文摘要字数在800~1000字为宜(限一页),英文摘要置于中文摘要下一页,内容应与中文摘要意同。
	为了便于文献检索,应在有摘要的一页下方另起一行注明论文的关键词(3~5个)。

  \keywords{学位论文;\LaTeX{};模板}
\end{abstract}

\begin{enabstract}
  An abstract of a dissertation is a summary and extraction of research work
  and contributions.
  Included in an abstract should be description of research topic and research
  objective, brief introduction to methodology and research process, and
  summarization of conclusion and contributions of the research.  An abstract
  should be characterized by independence and clarity and carry identical
  information with the dissertation.
  It should be such that the general idea and major contributions of the
  dissertation are conveyed without reading the dissertation.

  An abstract should be concise and to the point.
  It is a misunderstanding to make an abstract an outline of the dissertation
  and words ``the first chapter'', ``the second chapter'' and the like should
  be avoided in the abstract.

  Key words are terms used in a dissertation for indexing, reflecting core
  information of the dissertation.
  An abstract may contain a maximum of 5 key words, with semi-colons used in
  between to separate one another.

  \enkeywords{Dissertation; \LaTeX{}; template}
\end{enabstract}