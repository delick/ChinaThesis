% !TeX root = ../main.tex

\chapter{引用文献的标注}

\section{学校说明}
参考文献论文中按“作者,年”,如(邓起东,2000),见~\ref{sec:author-year}。文后按例编写并按拼音自动排列,文献作者姓名写到第三位,余者写“,等”或“,et al.”(此模板已符合要求)。
几种主要参考文献著录表的格式为:
\begin{enumerate}
\item \textbf{连续出版物}:序号作者.文题.刊名,年,卷号(期号):起--止页码。
\item \textbf{专(译)著}:序号作者.书名(,译者).出版地:出版者,出版年. 起--止页码。
\item \textbf{论文集}:序号作者.文题.见(in):编者,编(eds.).文集名.出版地:出版者,出版年. 起--止页码。
\item \textbf{学位论文}:序号姓名.文题:[XX学位论文].授予单位所在地:授予单位,授予年。
\item \textbf{专利}:序号申请者.专利名.国名,专利文献种类,专利号,出版日期。
\item \textbf{技术标准}:序号发布单位.技术标准代号.技术标准名称.出版地:出版者,出版日期。
\end{enumerate}

举例如下:
\begin{enumerate}
\item 王琪,丁国瑜,乔学军,等. 天山现今地壳快速缩短与南北地块的相对运动. 科学通报,2000,45(14):1538~1542
\item 邓起东,冯先岳,张培震,等. 天山活动构造. 北京:地震出版社,2000 
\item Dupont B. Bone marrow transplantation in severe combined immunodeficiency with an unrelated MLC compatible donor. In: White H J, Smith R, eds. Proceedings of the Third Annual Meeting of the International Society for Experimental Hematology. \item Houston: International Society for Experimental Hematology, 1974. 44~46
\item 任收麦. 阿尔金断裂运动学特征与活动时间约束:[博士学位论文]. 吉林:吉林大学,2002
\item 姜锡洲. 一种温热外敷药制备方法.中国专利,881056073,1980-07-26
\item 中华人民共和国国家技术监督局. GB3100~3102. 中华人民共和国国家标准-量与单位.北京:中国标准出版社,1994-11-01
\end{enumerate}

\section{顺序编码制}

模板使用 \pkg{natbib} 宏包来设置参考文献引用的格式,
更多引用方法可以参考该宏包的使用说明。

按照学校要求,正文的引用格式可用 \verb|\citep{knuth86a}| 代码实现,效果为:~\citep{knuth86a}。如需修改引用格式,请在 \texttt{main.tex} 文件中 frontmatter 之后(第 55 行)修改 \verb|\citestyle{authoryear}|,可选参数如下:

\subsection{角标数字标注法}

\citestyle{super}
\noindent
\begin{tabular}{l@{\quad$\Rightarrow$\quad}l}
  \verb|\cite{knuth86a}|         & \cite{knuth86a}         \\
  \verb|\citet{knuth86a}|        & \citet{knuth86a}        \\
  \verb|\cite[42]{knuth86a}|     & \cite[42]{knuth86a}     \\
  \verb|\cite{knuth86a,tlc2}|    & \cite{knuth86a,tlc2}    \\
  \verb|\cite{knuth86a,knuth84}| & \cite{knuth86a,knuth84} \\
\end{tabular}


\subsection{数字标注法}

\citestyle{numbers}
\noindent
\begin{tabular}{l@{\quad$\Rightarrow$\quad}l}
  \verb|\cite{knuth86a}|         & \cite{knuth86a}         \\
  \verb|\citet{knuth86a}|        & \citet{knuth86a}        \\
  \verb|\cite[42]{knuth86a}|     & \cite[42]{knuth86a}     \\
  \verb|\cite{knuth86a,tlc2}|    & \cite{knuth86a,tlc2}    \\
  \verb|\cite{knuth86a,knuth84}| & \cite{knuth86a,knuth84} \\
\end{tabular}



\subsection{著者-出版年制标注法}\label{sec:author-year}

\citestyle{authoryear}
\noindent
\begin{tabular}{l@{\quad$\Rightarrow$\quad}l}
  \verb|\cite{knuth86a}|         & \cite{knuth86a}         \\
  \verb|\citep{knuth86a}|        & \citep{knuth86a} $\surd$\\
  \verb|\cite[42]{knuth86a}|     & \cite[42]{knuth86a}     \\
  \verb|\cite{knuth86a,tlc2}|    & \cite{knuth86a,tlc2}    \\
  \verb|\cite{knuth86a,knuth84}| & \cite{knuth86a,knuth84} \\
\end{tabular}

\vskip 2ex \citestyle{super}
注意,参考文献列表中的每条文献在正文中都要被引用
\cite{slg,lyc,ljs,cgw,cjb,kqy,yhs,yx,dwx,jxz,wjk,syw,wf,xd,twh,huston}。
